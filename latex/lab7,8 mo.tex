
\documentclass[12pt, a4paper]{article}

\usepackage[utf8]{inputenc}
\usepackage[T1]{fontenc}
\usepackage[russian]{babel}
\usepackage{amsmath} 
\usepackage{amssymb} 
\usepackage{amsfonts} 
\usepackage[oglav,spisok,boldsect,eqwhole,figwhole,hyperref,hyperprint,remarks,greekit]{./style/fn2kursstyle}
\graphicspath{{./style/}{./figures/}}

\usepackage{multirow}
\usepackage{supertabular}
\usepackage{multicol}
% Параметры титульного листа
\title{}
\author{А.\,Д.~Егоров}
\supervisor{А.\,В.~Чередниченко}
\group{ФН2-52Б}
\date{2022}

% Переопределение команды \vec, чтобы векторы печатались полужирным курсивом
\renewcommand{\vec}[1]{\text{\mathversion{bold}${#1}$}}%{\bi{#1}}
\newcommand\thh[1]{\text{\mathversion{bold}${#1}$}}
%Переопределение команды нумерации перечней: точки заменяются на скобки
\renewcommand{\labelenumi}{\theenumi)}
\begin{document}

\maketitle

\tableofcontents



\newpage
\section{Лабораторная работа \textnumero 6}

\subsection {Постановка задачи}

В лабораторной работе необходимо найти с заданной точностью точку минимума и минимальное значение целевой функции. При исследовании для каждой функции брать два параметра точности поиска. Также для каждой функции и каждого параметра точности поиска взять две различные начальные точки. Выявить влияние на стоимость методов (количество вычисленных значений целевой функции)
\begin{itemize}
	\item параметров точности поиска;
	\item начальной точки;
	\item выпуклости;
	\item овражности функции (параметра $\alpha$ в функции Розенброка).
\end{itemize}
Используемые методы:
\begin{itemize}
	\item циклический покоординатный спуск;
	\item метод Хука-Дживса
	\item метод Розенброка.
\end{itemize}
Целевые функции:
\begin{itemize}
	\item $f_1(x,y)=5 x^2 + 4 x y + 2 y^2 + 4 \sqrt{5} (x + y) - 35$;
	\item  $f_2(x,y)=\left(x^2-y\right)^2+(x-1)^2$;
	\item  $f_3(x,y)=75\left(x^2-y\right)^2+(x-1)^2$.
\end{itemize}
Заданная точность: 
\begin{itemize}
	\item $\varepsilon=0.01$;
	\item $\varepsilon=0.000001$.
\end{itemize}

\newpage

\subsection {Тестовые примеры и результаты расчетов}



\begin{table}[!h]
	\centering
	\begin{tabular}[c]{|c|c|c|c|c|}
		\hline
		\multirow{3}{4.5cm}{Входные данные}& \multirow{3}{2.8cm}{Точка минимума}& \multirow{3}{2.5cm}{Наименьшее значение функции} & \multirow{3}{2.3cm}{Количество итераций}   & \multirow{3}{2.6cm}{Количество вычисленных функций}   \\ &&&& \\ &&&& \\ 
		
		\hline
		\multirow{3}{5.5cm}{Квадратичная функция, Метод ПС, \newline $\varepsilon=0.01$, $[x,y]=[-1,-2]$}& \multirow{3}{2.8cm}{$
			x=-2.23$\\$ y= -4.45$}& \multirow{3}{2.5cm}{$-28$} & \multirow{3}{2.3cm}{19}   & \multirow{3}{2.6cm}{22}   \\ &&&& \\ &&&& \\ \hline
		
		\multirow{3}{5.5cm}{Квадратичная функция, Метод Хука-Дживса, \newline  $\varepsilon=0.000001$, $[x,y]=[-2,1]$}&\multirow{3}{2.8cm}{$
			x=-2.23607$\\$ y= -4.47213$}& \multirow{3}{2.5cm}{-28} & \multirow{3}{2.3cm}{32}   & \multirow{3}{2.6cm}{28}   \\ &&&& \\ &&&& \\ \hline
		
		\multirow{3}{5.4cm}{Функция Розенброка $\alpha=1$, $\varepsilon=0.01$, $[x,y]=[-2,1]$}&\multirow{3}{2.8cm}{$
			x=0.98$\\$ y= 0.96$}& \multirow{3}{2.5cm}{0} & \multirow{3}{2.3cm}{21}   & \multirow{3}{2.6cm}{26}   \\ &&&& \\ &&&& \\ \hline
		
		\multirow{3}{4.4cm}{Функция Розенброка $\alpha=1$, $\varepsilon=0.000001$, $[x,y]=[-2,1]$}& \multirow{3}{2.6cm}{$
			x=0.999994$\\$ y= 0.999987$}& \multirow{3}{2.5cm}{0} & \multirow{3}{2.3cm}{180}   & \multirow{3}{2.6cm}{318}   \\ &&&& \\ &&&& \\ \hline
		
		\multirow{3}{4.4cm}{Функция Розенброка $\alpha=75$, $\varepsilon=0.01$, $[x,y]=[-2,1]$}& \multirow{3}{2.8cm}{$
			x=0.66$\\$ y= 0,44$}& \multirow{3}{2.5cm}{0.1} & \multirow{3}{2.3cm}{19}   & \multirow{3}{2.6cm}{22}   \\ &&&& \\ &&&& \\ \hline
		
		\multirow{3}{4cm}{Функция Розенброка $\alpha=75$, $\varepsilon=0.000001$, $[x,y]=[-2,1]$}& \multirow{3}{2.8cm}{$
			x=0.999459$\\$ y= 0.998917$}& \multirow{3}{2.5cm}{0} & \multirow{3}{2.3cm}{11304}   & \multirow{3}{2.6cm}{22566}   \\ &&&& \\ &&&& \\ [1ex]  \hline  
		
	\end{tabular}
	\centering{}
	\caption{\centering{Результаты вычислений для регулярного симплекса при начальной длине ребра $l=2$ и коэффиценте редукции $\delta=\frac1 2$ в зависимости от функции и точности }}
\end{table}

\newpage
\section{Лабораторная работа \textnumero 7}


\subsection {Постановка задачи}

В лабораторной работе необходимо найти с заданной точностью точку минимума и минимальное значение целевой функции. При исследовании для каждой функции брать два параметра точности поиска. Также для каждой функции и каждого параметра точности поиска взять две различные начальные точки. Выявить влияние на стоимость методов (количество вычисленных значений целевой функции)
\begin{itemize}
	\item параметров точности поиска;
	\item начальной точки;
	\item выпуклости;
	\item овражности функции (параметра $\alpha$ в функции Розенброка).
\end{itemize}
Используемые методы:
\begin{itemize}
	\item регулярный симплекс;
	\item нерегулярный симплекс (метод Нелдера-Мида).
\end{itemize}
Целевые функции:
\begin{itemize}
	\item $f_1(x,y)=6 x^2-4 x y+4 \sqrt{5} (x+2 y)+3 y^2+22$;
	\item  $f_2(x,y)=\left(x^2-y\right)^2+(x-1)^2$;
	\item  $f_3(x,y)=75\left(x^2-y\right)^2+(x-1)^2$.
\end{itemize}
Заданная точность: 
\begin{itemize}
	\item $\varepsilon=0.01$;
	\item $\varepsilon=0.000001$.
\end{itemize}
\newpage

\subsection {Тестовые примеры и результаты расчетов}
\begin{table}[!h]
	\centering
\begin{tabular}[c]{|c|c|c|c|c|}
	\hline
	  \multirow{3}{4.5cm}{Входные данные}& \multirow{3}{2.8cm}{Точка минимума}& \multirow{3}{2.5cm}{Наименьшее значение функции} & \multirow{3}{2.3cm}{Количество итераций}   & \multirow{3}{2.6cm}{Количество вычисленных функций}   \\ &&&& \\ &&&& \\ \hline
	    \multirow{3}{4.5cm}{Квадратичная функция, $\varepsilon=0.01$, $[x,y]=[-2,1]$}& \multirow{3}{2.8cm}{$
	    		x=-2.23$\\$ y= -4.45$}& \multirow{3}{2.5cm}{$-28$} & \multirow{3}{2.3cm}{19}   & \multirow{3}{2.6cm}{22}   \\ &&&& \\ &&&& \\ \hline
	    
	      \multirow{3}{4.5cm}{Квадратичная функция,  $\varepsilon=0.000001$, $[x,y]=[-2,1]$}&\multirow{3}{2.8cm}{$
	      	x=-2.23607$\\$ y= -4.47213$}& \multirow{3}{2.5cm}{-28} & \multirow{3}{2.3cm}{32}   & \multirow{3}{2.6cm}{28}   \\ &&&& \\ &&&& \\ \hline
	      
	        \multirow{3}{4.4cm}{Функция Розенброка $\alpha=1$, $\varepsilon=0.01$, $[x,y]=[-2,1]$}&\multirow{3}{2.8cm}{$
	        	x=0.98$\\$ y= 0.96$}& \multirow{3}{2.5cm}{0} & \multirow{3}{2.3cm}{21}   & \multirow{3}{2.6cm}{26}   \\ &&&& \\ &&&& \\ \hline
	        
	          \multirow{3}{4.4cm}{Функция Розенброка $\alpha=1$, $\varepsilon=0.000001$, $[x,y]=[-2,1]$}& \multirow{3}{2.6cm}{$
	          	x=0.999994$\\$ y= 0.999987$}& \multirow{3}{2.5cm}{0} & \multirow{3}{2.3cm}{180}   & \multirow{3}{2.6cm}{318}   \\ &&&& \\ &&&& \\ \hline
	          
	           \multirow{3}{4.4cm}{Функция Розенброка $\alpha=75$, $\varepsilon=0.01$, $[x,y]=[-2,1]$}& \multirow{3}{2.8cm}{$
	           	x=0.66$\\$ y= 0,44$}& \multirow{3}{2.5cm}{0.1} & \multirow{3}{2.3cm}{19}   & \multirow{3}{2.6cm}{22}   \\ &&&& \\ &&&& \\ \hline
	           
	           \multirow{3}{4cm}{Функция Розенброка $\alpha=75$, $\varepsilon=0.000001$, $[x,y]=[-2,1]$}& \multirow{3}{2.8cm}{$
	           	x=0.999459$\\$ y= 0.998917$}& \multirow{3}{2.5cm}{0} & \multirow{3}{2.3cm}{11304}   & \multirow{3}{2.6cm}{22566}   \\ &&&& \\ &&&& \\ [1ex]  \hline  

\end{tabular}
\centering{}
	\caption{\centering{Результаты вычислений для регулярного симплекса при начальной длине ребра $l=2$ и коэффиценте редукции $\delta=\frac1 2$ в зависимости от функции и точности }}
\end{table}
\begin{table}[!h]
	\centering
	\begin{tabular}[c]{|c|c|c|c|c|}
		\hline
		\multirow{3}{4.5cm}{Входные данные}& \multirow{3}{2.8cm}{Точка минимума}& \multirow{3}{2.5cm}{Наименьшее значение функции} & \multirow{3}{2.3cm}{Количество итераций}   & \multirow{3}{2.6cm}{Количество вычисленных функций}   \\ &&&& \\ &&&& \\ \hline
		\multirow{3}{4.5cm}{Квадратичная функция, $\varepsilon=0.01$, $[x,y]=[-2,1]$}& \multirow{3}{2.8cm}{$
			x=-2.23$\\$ y= -4.45$}& \multirow{3}{2.5cm}{$-28$} & \multirow{3}{2.3cm}{18}   & \multirow{3}{2.6cm}{96}   \\ &&&& \\ &&&& \\ \hline
		
		\multirow{3}{4.5cm}{Квадратичная функция,  $\varepsilon=0.000001$, $[x,y]=[-2,1]$}&\multirow{3}{2.8cm}{$
			x=-2.23609$\\$ y= -4.47311$}& \multirow{3}{2.5cm}{-28} & \multirow{3}{2.3cm}{30}   & \multirow{3}{2.6cm}{162}   \\ &&&& \\ &&&& \\ \hline
		
		\multirow{3}{4.4cm}{Функция Розенброка $\alpha=1$, $\varepsilon=0.01$, $[x,y]=[-2,1]$}&\multirow{3}{2.8cm}{$
			x=1.01$\\$ y= 1.01$}& \multirow{3}{2.5cm}{0} & \multirow{3}{2.3cm}{15}   & \multirow{3}{2.6cm}{84}   \\ &&&& \\ &&&& \\ \hline
		
		\multirow{3}{4.4cm}{Функция Розенброка $\alpha=1$, $\varepsilon=0.000001$, $[x,y]=[-2,1]$}& \multirow{3}{2.6cm}{$
			x=1.00026$\\$ y= 1.00115$}& \multirow{3}{2.5cm}{0} & \multirow{3}{2.3cm}{36}   & \multirow{3}{2.6cm}{192}   \\ &&&& \\ &&&& \\ \hline
		
		\multirow{3}{4.4cm}{Функция Розенброка $\alpha=75$, $\varepsilon=0.01$, $[x,y]=[-2,1]$}& \multirow{3}{2.8cm}{$
			x=0.75$\\$ y= 0,56$}& \multirow{3}{2.5cm}{0.06} & \multirow{3}{2.3cm}{55}   & \multirow{3}{2.6cm}{294}   \\ &&&& \\ &&&& \\ \hline
		
		\multirow{3}{4cm}{Функция Розенброка $\alpha=75$, $\varepsilon=0.000001$, $[x,y]=[-2,1]$}& \multirow{3}{2.8cm}{$
			x=0.981937$\\$ y= 0.964099$}& \multirow{3}{2.5cm}{0.0} & \multirow{3}{2.3cm}{124}   & \multirow{3}{2.6cm}{654}   \\ &&&& \\ &&&& \\ [1ex]  \hline  
		
	\end{tabular}
	\centering{}
	\caption{\centering{Результаты вычислений для нерегулярного симплекса при начальной длине ребра $l=2$, коэффицентах отражения $\alpha=1$, растяжения $\beta=2$, сжатия $\gamma=1/2$, редукции $\delta=\frac1 2$ в зависимости от функции и точности }}
\end{table}
	
\newpage

\begin{table}[!h]
	\centering
	\begin{tabular}[c]{|c|c|c|c|c|c|}
		\hline
		\multirow{2}{2.8cm}{Начальная точка}& \multirow{2}{2.25cm}{$[-2,1]$}&  \multirow{2}{2,25cm}{$[-2,10]$}&
		\multirow{2}{2,25cm}{$[-10,1]$}&\multirow{2}{2,25cm}{$[10,10]$} &
		\multirow{2}{2,25cm}{$[0,0]$}
		\\ & & & & &  \\  \hline  & \multicolumn{5}{c|}{Метод регулярного сиплекса} \\ \hline  
		\multirow{3}{2.8cm}{Количество итераций}& \multirow{3}{2,25cm}{21}&  \multirow{3}{2,25cm}{335}&
		\multirow{3}{2,25cm}{41}&\multirow{3}{2,25cm}{544} &
		\multirow{3}{2,25cm}{22}
		\\ & & & & & \\ & & & & &   \\ \hline
		\multirow{3}{2.8cm}{Количество вычисленых функций}& \multirow{3}{2,25cm}{26}&  \multirow{3}{2,25cm}{654}&
		\multirow{3}{2,25cm}{66}&\multirow{3}{2,25cm}{1072} &
		\multirow{3}{2,25cm}{28}
		\\ & & & & &  	\\ & & & & &\\ \hline
		  &
		\multicolumn{5}{c|}{Метод нерегулярного сиплекса} \\ \hline  
			\multirow{3}{2.8cm}{Количество итераций}& \multirow{3}{2,25cm}{15}&  \multirow{3}{2,25cm}{27}&
			\multirow{3}{2,25cm}{37}&\multirow{3}{2,25cm}{35} &
			\multirow{3}{2,25cm}{10}
		\\ & & & & & \\ & & & & &   \\ \hline
		\multirow{3}{2.8cm}{Количество вычисленых функций}& \multirow{3}{2,25cm}{84}&  \multirow{3}{2,25cm}{144}&
		\multirow{3}{2,25cm}{196}&\multirow{3}{2,25cm}{184} &
		\multirow{3}{2,25cm}{54}
		\\ & & & & &  	\\ & & & & &\\ \hline
	
		
	\end{tabular}
	\centering{}
	\caption{\centering{Результаты вычислений для функции Розенброка $f_2(x,y)$ для регулярного симплекса и нерегулярного симплекса при начальной длине ребра $l=2$, коэффицентах отражения $\alpha=1$, растяжения $\beta=2$, сжатия $\gamma=1/2$, редукции $\delta=\frac1 2$ в зависимости от начальной точки}}
\end{table}




\newpage 
\begin{figure}[!h]
	\begin{minipage}[h]{140pt}
		%\center{\includegraphics[width=140pt]{simplex1.pdf}} \\а) 
	\end{minipage}
	\hfill
	\begin{minipage}[h]{140pt}
		%\center{\includegraphics[width=140pt]{simplex5.pdf}} \\б)
	\end{minipage}
	\hfill
	\begin{minipage}[h]{140pt}
		%\center{\includegraphics[width=140pt]{simplex9.pdf}} \\в) 
	\end{minipage}
	\vfill
	$\,$
	\centering{ }
	\caption{\centering{Визуализация метода регулярного симплекса при $\varepsilon=0.01$ для а) квадратичной функции $f_1(x,y)$, б) функции Розенброка $f_2(x,y)$ в) функции Розенброка $f_3(x,y)$}}
\end{figure}

\begin{figure}[!h]
		\hfill
	\begin{minipage}[h]{210pt}
		%\center{\includegraphics[width=140pt]{simplex3.pdf}} \\а) 
	\end{minipage}
	\hfill
	\begin{minipage}[h]{210pt}
	%	\center{\includegraphics[width=140pt]{simplex7.pdf}} \\б)
	\end{minipage}
	\hfill
	\vfill
	$\,$
	\centering{ }
	\caption{\centering{Визуализация метода регулярного симплекса при $\varepsilon=0.000001$ для а) квадратичной функции $f_1(x,y)$, б) функции Розенброка $f_2(x,y)$}}
\end{figure}

\begin{figure}[!h]
	\begin{minipage}[h]{140pt}
	%	\center{\includegraphics[width=140pt]{simplex2.pdf}} \\а) 
	\end{minipage}
	\hfill
	\begin{minipage}[h]{140pt}
	%	\center{\includegraphics[width=140pt]{simplex6.pdf}} \\б)
	\end{minipage}
	\hfill
	\begin{minipage}[h]{140pt}
	%	\center{\includegraphics[width=140pt]{simplex10.pdf}} \\в) 
	\end{minipage}
	\vfill
	$\,$
	\centering{ }
	\caption{\centering{Визуализация метода нерегулярного симплекса при $\varepsilon=0.01$ для а) квадратичной функции $f_1(x,y)$, б) функции Розенброка $f_2(x,y)$ в) функции Розенброка $f_3(x,y)$}}
\end{figure}

\begin{figure}[!h]
	\begin{minipage}[h]{140pt}
	%	\center{\includegraphics[width=140pt]{simplex4.pdf}} \\а) 
	\end{minipage}
	\hfill
	\begin{minipage}[h]{140pt}
	%	\center{\includegraphics[width=140pt]{simplex8.pdf}} \\б)
	\end{minipage}
\hfill
\begin{minipage}[h]{140pt}
%	\center{\includegraphics[width=140pt]{simplex12.pdf}} \\в) 
\end{minipage}
	\vfill
	$\,$
	\centering{ }
	\caption{\centering{Визуализация метода нерегулярного симплекса при $\varepsilon=0.00001$ для а) квадратичной функции $f_1(x,y)$, б) функции Розенброка $f_2(x,y)$}}
\end{figure}


\clearpage
\subsection {Выводы}

В результате выполнения лабораторной работы были реализованы два метода:
\begin{itemize}
	\item Регулярный симплекс,
	\item Нерегулярный симплекс (метод Нелдера-Мида).
\end{itemize}

Во всех методах с заранее заданной точностью были получены точка минимума и минимальное значение в этой точке. 

При поиске точки минимума для квадратичной функции оба методы показывают хорошие результаты, но эффективнее оказался поиск с помощью регулярного симплекса, так как требовал меньшего количества вычислений функции, так как случае нерегулярного симплекса много вычислений уходит на одномерную минимизацию.
При поиске точки минимума для функции Розенброка лучшие результаты у
метода нерегулярного симплекса: метод регулярного симплекса требовал меньшего вычисления функций при малой точности, но при увеличении точности гораздо эффективнее было использование нерегулярного симплекса. 

 К плюсами данных методов можно отнести то, что для их реализации не требуется находить градиенты или матрицы Гесса, а их поиск, в свою очередь, порой является весьма нетривиальной задачей. 

\newpage
\section{Лабораторная работа \textnumero 8}


\subsection {Постановка задачи}
В лабораторной работе необходимо найти с заданной точностью точку минимума, принадлежащей заданному допустимому множеству, и минимальное значение целевой функции в ней. При исследовании для каждой функции брать два параметра точности поиска. Также для каждой функции и каждого параметра точности поиска взять две различные начальные точки. Выявить влияние на стоимость методов (количество вычисленных значений целевой функции)
\begin{itemize}
	\item параметров точности поиска;
	\item начальной точки;
	\item выпуклости;
	\item овражности функции (параметра $\alpha$ в функции Розенброка).
\end{itemize}
Используемые методы:
\begin{itemize}
	\item метод внутренних штрафных функций (барьерных функций);
	\item метод внешних штрафных функций.
\end{itemize}
Целевые функции:
\begin{itemize}
	\item $f_1(x,y)=6 x^2-4 x y+4 \sqrt{5} (x+2 y)+3 y^2+22$;
	\item  $f_2(x,y)=\left(x^2-y\right)^2+(x-1)^2$;
	\item  $f_3(x,y)=75\left(x^2-y\right)^2+(x-1)^2$.
\end{itemize}
Заданная точность: 
\begin{itemize}
	\item $\varepsilon=0.01$;
	\item $\varepsilon=0.000001$.
\end{itemize}
Заданное допустимое множество: 
\begin{itemize}
	\item $A: \quad x\leq0,\quad y\leq0,\quad x+y\ge 10$;
	\item $B: \quad \frac{(x+3)^2}{4}+\frac{(y+4)^2}{9} \leq10$.
\end{itemize}
\newpage
\subsection {Тестовые примеры и результаты расчетов}

\begin{table}[!h]
	\centering
	\begin{tabular}[c]{|c|c|c|c|c|c|c|}
		\hline
		\multirow{3}{3cm}{Входные данные}& \multirow{3}{2.8cm}{Точка min.}& \multirow{3}{1.8cm}{Min. значение} & \multirow{3}{1.6cm}{Кол. итераций}& \multicolumn{3}{c|}{{Колличество вычисленний}}    \\ \cline{5-7} &&&&\multirow{2}{1.7cm}{функций}&\multirow{2}{2cm}{градиентов}&\multirow{2}{1.7cm}{матриц Гессе} \\ &&&&&& \\ \hline
		\multirow{3}{3.1cm}{Квад. ф., $\varepsilon=0.01$, $[x,y]=[-2,2]$}& \multirow{3}{2.8cm}{$
			x=0.002$\\$ y= 0.001$}& \multirow{3}{1.8cm}{22.06} & \multirow{3}{1.6cm}{20}   & \multirow{3}{1.7cm}{21}&\multirow{3}{2cm}{97}&\multirow{3}{1.7cm}{97}   \\ &&&&&& \\ &&&&&& \\ \hline
		\multirow{3}{3cm}{Квад. ф., $\varepsilon=0.000001$, $[x,y]=[-2,2]$}& \multirow{3}{2.8cm}{$
			x=0.002608$\\$ y= 0.001847$}& \multirow{3}{1.8cm}{22.5646} & \multirow{3}{1.6cm}{20}   & \multirow{3}{1.7cm}{21}&\multirow{3}{2cm}{118}&\multirow{3}{1.7cm}{118}   \\ &&&&&& \\ &&&&&& \\ \hline
		\multirow{3}{3cm}{Ф. Розенброка $1$ , $\varepsilon=0.01$, $[x,y]=[-2,2]$}& \multirow{3}{2.8cm}{$
			x=1.01$\\$ y= 1.02$}& \multirow{3}{1.8cm}{0} & \multirow{3}{1.6cm}{14}   & \multirow{3}{1.7cm}{15}&\multirow{3}{2cm}{49}&\multirow{3}{1.7cm}{49}   \\ &&&&&& \\ &&&&&& \\ \hline
		\multirow{3}{3cm}{Ф. Розенброка $1$, $\varepsilon=0.000001$, $[x,y]=[-2,2]$}& \multirow{3}{2.8cm}{$
			x=1.00009$\\$ y= 1.00021$}& \multirow{3}{1.8cm}{0} & \multirow{3}{1.6cm}{20}   & \multirow{3}{1.7cm}{21}&\multirow{3}{2cm}{77}&\multirow{3}{1.7cm}{77}   \\ &&&&&& \\ &&&&&& \\ \hline
		\multirow{3}{3cm}{Ф. Розенброка $75$, $\varepsilon=0.01$, $[x,y]=[-2,2]$}& \multirow{3}{2.8cm}{$
			x=1.01$\\$ y= 1.02$}& \multirow{3}{1.8cm}{0} & \multirow{3}{1.6cm}{14}   & \multirow{3}{1.7cm}{15}&\multirow{3}{2cm}{82}&\multirow{3}{1.7cm}{82}   \\ &&&&&& \\ &&&&&& \\ \hline
		\multirow{3}{3.1cm}{Ф. Розенброка $75$, $\varepsilon=0.000001$, $[x,y]=[-2,2]$}& \multirow{3}{2.8cm}{$
			x=1.00009$\\$ y= 1.00018$}& \multirow{3}{1.8cm}{0} & \multirow{3}{1.6cm}{20}   & \multirow{3}{1.7cm}{21}&\multirow{3}{2cm}{124}&\multirow{3}{1.7cm}{124}   \\\ &&&&&& \\ &&&&&& \\ \hline
	\end{tabular}
	\centering{}
	\caption{\centering{Результаты вычислений для метода внутренних штрафных функций и для допустимого множества $A$ в зависимости от функции и точности  }}
\end{table}

\begin{table}[!h]
	\centering
	\begin{tabular}[c]{|c|c|c|c|c|c|c|}
		\hline
\multirow{3}{3cm}{Входные данные}& \multirow{3}{2.8cm}{Точка min.}& \multirow{3}{1.8cm}{Min. значение} & \multirow{3}{1.6cm}{Кол. итераций}& \multicolumn{3}{c|}{{Колличество вычисленний}}    \\ \cline{5-7} &&&&\multirow{2}{1.7cm}{функций}&\multirow{2}{2cm}{градиентов}&\multirow{2}{1.7cm}{матриц Гессе} \\ &&&&&& \\ \hline
	\multirow{3}{3.1cm}{Квад. ф., $\varepsilon=0.01$, $[x,y]=[-2,2]$}& \multirow{3}{2.8cm}{$
		x=-2.23$\\$ y= -4.47$}& \multirow{3}{1.8cm}{-28} & \multirow{3}{1.6cm}{10}   & \multirow{3}{1.7cm}{11}&\multirow{3}{2cm}{6}&\multirow{3}{1.7cm}{6}   \\ &&&&&& \\ &&&&&& \\ \hline
		\multirow{3}{3cm}{Квад. ф., $\varepsilon=0.000001$, $[x,y]=[-2,2]$}& \multirow{3}{2.8cm}{$
		x=-2.23607$\\$ y= 4.47214$}& \multirow{3}{1.8cm}{-28} & \multirow{3}{1.6cm}{20}   & \multirow{3}{1.7cm}{21}&\multirow{3}{2cm}{20}&\multirow{3}{1.7cm}{20}   \\ &&&&&& \\ &&&&&& \\ \hline
		\multirow{3}{3cm}{Ф. Розенброка $1$ , $\varepsilon=0.01$, $[x,y]=[-2,2]$}& \multirow{3}{2.8cm}{$
		x=0.98$\\$ y= 0.97$}& \multirow{3}{1.8cm}{0} & \multirow{3}{1.6cm}{11}   & \multirow{3}{1.7cm}{12}&\multirow{3}{2cm}{19}&\multirow{3}{1.7cm}{19}   \\ &&&&&& \\ &&&&&& \\ \hline
		\multirow{3}{3cm}{Ф. Розенброка $1$, $\varepsilon=0.000001$, $[x,y]=[-2,2]$}& \multirow{3}{2.8cm}{$
		x=0.999988$\\$ y= 0.999972$}& \multirow{3}{1.8cm}{0} & \multirow{3}{1.6cm}{20}   & \multirow{3}{1.7cm}{21}&\multirow{3}{2cm}{44}&\multirow{3}{1.7cm}{44}   \\ &&&&&& \\ &&&&&& \\ \hline
		\multirow{3}{3cm}{Ф. Розенброка $75$, $\varepsilon=0.01$, $[x,y]=[-2,2]$}& \multirow{3}{2.8cm}{$
		x=0.98$\\$ y= 0.97$}& \multirow{3}{1.8cm}{0} & \multirow{3}{1.6cm}{11}   & \multirow{3}{1.7cm}{12}&\multirow{3}{2cm}{33}&\multirow{3}{1.7cm}{33}   \\ &&&&&& \\ &&&&&& \\ \hline
		\multirow{3}{3.1cm}{Ф. Розенброка $75$, $\varepsilon=0.000001$, $[x,y]=[-2,2]$}& \multirow{3}{2.8cm}{$
		x=0.999988$\\$ y= 0.999975$}& \multirow{3}{1.8cm}{0} & \multirow{3}{1.6cm}{20}   & \multirow{3}{1.7cm}{21}&\multirow{3}{2cm}{64}&\multirow{3}{1.7cm}{64}   \\\ &&&&&& \\ &&&&&& \\ \hline
	\end{tabular}
	\centering{}
	\caption{\centering{Результаты вычислений для метода внутренних штрафных функций и для допустимого множества $B$ в зависимости от функции и точности  }}
\end{table}

\begin{table}[!h]
	\centering
	\begin{tabular}[c]{|c|c|c|c|c|c|c|}
		\hline
		\multirow{3}{3cm}{Входные данные}& \multirow{3}{2.8cm}{Точка min.}& \multirow{3}{1.8cm}{Min. значение} & \multirow{3}{1.6cm}{Кол. итераций}& \multicolumn{3}{c|}{{Колличество вычисленний}}    \\ \cline{5-7} &&&&\multirow{2}{1.7cm}{функций}&\multirow{2}{2cm}{градиентов}&\multirow{2}{1.7cm}{матриц Гессе} \\ &&&&&& \\ \hline
		\multirow{3}{3.1cm}{Квад. ф., $\varepsilon=0.01$, $[x,y]=[-2,2]$}& \multirow{3}{2.8cm}{$
			x=-0.0002$\\$ y= -0.0005$}& \multirow{3}{1.8cm}{21.98} & \multirow{3}{1.6cm}{10}   & \multirow{3}{1.7cm}{11}&\multirow{3}{2cm}{11}&\multirow{3}{1.7cm}{11}   \\ &&&&&& \\ &&&&&& \\ \hline
		\multirow{3}{3cm}{Квад. ф., $\varepsilon=0.000001$, $[x,y]=[-2,2]$}& \multirow{3}{2.8cm}{$
			x=0$\\$ y= 0$}& \multirow{3}{1.8cm}{22} & \multirow{3}{1.6cm}{23}   & \multirow{3}{1.7cm}{24}&\multirow{3}{2cm}{24}&\multirow{3}{1.7cm}{24}   \\ &&&&&& \\ &&&&&& \\ \hline
		\multirow{3}{3cm}{Ф. Розенброка $1$ , $\varepsilon=0.01$, $[x,y]=[-2,2]$}& \multirow{3}{2.8cm}{$
			x=0.99$\\$ y= 0.99$}& \multirow{3}{1.8cm}{0} & \multirow{3}{1.6cm}{2}   & \multirow{3}{1.7cm}{3}&\multirow{3}{2cm}{5}&\multirow{3}{1.7cm}{5}   \\ &&&&&& \\ &&&&&& \\ \hline
		\multirow{3}{3cm}{Ф. Розенброка $1$, $\varepsilon=0.000001$, $[x,y]=[-2,2]$}& \multirow{3}{2.8cm}{$
			x=1$\\$ y= 1$}& \multirow{3}{1.8cm}{0} & \multirow{3}{1.6cm}{2}   & \multirow{3}{1.7cm}{3}&\multirow{3}{2cm}{7}&\multirow{3}{1.7cm}{7}   \\ &&&&&& \\ &&&&&& \\ \hline
		\multirow{3}{3cm}{Ф. Розенброка $75$, $\varepsilon=0.01$, $[x,y]=[-2,2]$}& \multirow{3}{2.8cm}{$
			x=0.99$\\$ y= 0.99$}& \multirow{3}{1.8cm}{0} & \multirow{3}{1.6cm}{2}   & \multirow{3}{1.7cm}{3}&\multirow{3}{2cm}{11}&\multirow{3}{1.7cm}{11}   \\ &&&&&& \\ &&&&&& \\ \hline
		\multirow{3}{3.1cm}{Ф. Розенброка $75$, $\varepsilon=0.000001$, $[x,y]=[-2,2]$}& \multirow{3}{2.8cm}{$
			x=1$\\$ y= 1$}& \multirow{3}{1.8cm}{0} & \multirow{3}{1.6cm}{2}   & \multirow{3}{1.7cm}{3}&\multirow{3}{2cm}{12}&\multirow{3}{1.7cm}{12}   \\\ &&&&&& \\ &&&&&& \\ \hline
	\end{tabular}
	\centering{}
	\caption{\centering{Результаты вычислений для метода внешних штрафных функций и для допустимого множества $A$ в зависимости от функции и точности  }}
\end{table}

\begin{table}[!h]
	\centering
	\begin{tabular}[c]{|c|c|c|c|c|c|c|}
		\hline
		\multirow{3}{3cm}{Входные данные}& \multirow{3}{2.8cm}{Точка min.}& \multirow{3}{1.8cm}{Min. значение} & \multirow{3}{1.6cm}{Кол. итераций}& \multicolumn{3}{c|}{{Колличество вычисленний}}    \\ \cline{5-7} &&&&\multirow{2}{1.7cm}{функций}&\multirow{2}{2cm}{градиентов}&\multirow{2}{1.7cm}{матриц Гессе} \\ &&&&&& \\ \hline
		\multirow{3}{3.1cm}{Квад. ф., $\varepsilon=0.01$, $[x,y]=[-2,2]$}& \multirow{3}{2.8cm}{$
			x=-2.23$\\$ y= -4.47$}& \multirow{3}{1.8cm}{-28} & \multirow{3}{1.6cm}{2}   & \multirow{3}{1.7cm}{21}&\multirow{3}{2cm}{1}&\multirow{3}{1.7cm}{1}   \\ &&&&&& \\ &&&&&& \\ \hline
		\multirow{3}{3cm}{Квад. ф., $\varepsilon=0.000001$, $[x,y]=[-2,2]$}& \multirow{3}{2.8cm}{$
			x=-2.23607$\\$ y= -4.447214$}& \multirow{3}{1.8cm}{-28} & \multirow{3}{1.6cm}{2}   & \multirow{3}{1.7cm}{3}&\multirow{3}{2cm}{1}&\multirow{3}{1.7cm}{1}   \\\ &&&&&& \\ &&&&&& \\ \hline
		\multirow{3}{3cm}{Ф. Розенброка $1$ , $\varepsilon=0.01$, $[x,y]=[-2,2]$}& \multirow{3}{2.8cm}{$
			x=1$\\$ y= 1$}& \multirow{3}{1.8cm}{0} & \multirow{3}{1.6cm}{2}   & \multirow{3}{1.7cm}{3}&\multirow{3}{2cm}{6}&\multirow{3}{1.7cm}{6}   \\\ &&&&&& \\ &&&&&& \\ \hline
		\multirow{3}{3cm}{Ф. Розенброка $1$, $\varepsilon=0.000001$, $[x,y]=[-2,2]$}& \multirow{3}{2.8cm}{$
			x=1$\\$ y= 1$}& \multirow{3}{1.8cm}{0} & \multirow{3}{1.6cm}{2}   & \multirow{3}{1.7cm}{3}&\multirow{3}{2cm}{7}&\multirow{3}{1.7cm}{7}   \\\ &&&&&& \\ &&&&&& \\ \hline
		\multirow{3}{3cm}{Ф. Розенброка $75$, $\varepsilon=0.01$, $[x,y]=[-2,2]$}& \multirow{3}{2.8cm}{$
			x=1$\\$ y= 1$}& \multirow{3}{1.8cm}{0} & \multirow{3}{1.6cm}{2}   & \multirow{3}{1.7cm}{3}&\multirow{3}{2cm}{5}&\multirow{3}{1.7cm}{5}   \\\ &&&&&& \\ &&&&&& \\ \hline
		\multirow{3}{3.1cm}{Ф. Розенброка $75$, $\varepsilon=0.000001$, $[x,y]=[-2,2]$}& \multirow{3}{2.8cm}{$
			x=1$\\$ y= 1$}& \multirow{3}{1.8cm}{0} & \multirow{3}{1.6cm}{2}   & \multirow{3}{1.7cm}{3}&\multirow{3}{2cm}{5}&\multirow{3}{1.7cm}{5}   \\\ &&&&&& \\ &&&&&& \\ \hline
	\end{tabular}
	\centering{}
	\caption{\centering{Результаты вычислений для метода внешних штрафных функций и для допустимого множества $B$ в зависимости от функции и точности  }}
\end{table}

\begin{table}[!h]
	\centering
	\begin{tabular}[c]{|c|c|c|c|}
		\hline
		\multirow{2}{2.8cm}{Начальная точка}& \multirow{2}{2.7cm}{$[-2,2]$}&  \multirow{2}{2,7cm}{$[2,4]$}&
		\multirow{2}{2,7cm}{$[2,2]$}
		\\ & & &   \\  \hline Количество & \multicolumn{3}{c|}{Метод внутренних штрафных функций на $A$} \\ \hline  
		\multirow{2}{2.8cm}{Итераций}& \multirow{2}{2,25cm}{14}&  \multirow{2}{2,25cm}{14}&
		\multirow{2}{2,25cm}{14}
		\\ & & &  \\ \hline
		\multirow{2}{2.8cm}{Вычисленных функций}& \multirow{2}{2,25cm}{15}&  \multirow{2}{2,25cm}{15}&
		\multirow{2}{2,25cm}{15}
		\\ & & &   \\ \hline
		\multirow{2}{2.8cm}{Вычисленных градиентов}& \multirow{2}{2,25cm}{49}&  \multirow{2}{2,25cm}{23}&
		\multirow{2}{2,25cm}{25}
		\\ & & &    \\ \hline
		\multirow{2}{2.8cm}{Вычисленных матриц Гессе}& \multirow{2}{2,25cm}{49}&  \multirow{2}{2,25cm}{23}&
		\multirow{2}{2,25cm}{25}
		\\ & & &   \\ 
	\hline Количество & \multicolumn{3}{c|}{Метод внeшних штрафных функций на $A$} \\ \hline  
	\multirow{2}{2.8cm}{Итераций}& \multirow{2}{2,25cm}{2}&  \multirow{2}{2,25cm}{2}&
	\multirow{2}{2,25cm}{2}
	\\ & & &   \\ \hline
	\multirow{2}{2.8cm}{Вычисленных функций}& \multirow{2}{2,25cm}{3}&  \multirow{2}{2,25cm}{3}&
	\multirow{2}{2,25cm}{3}
	\\ & & &   \\ \hline
	\multirow{2}{2.8cm}{Вычисленных градиентов}& \multirow{2}{2,25cm}{5}&  \multirow{2}{2,25cm}{2}&
	\multirow{2}{2,25cm}{4}
	\\ & & &   \\ \hline
	\multirow{2}{2.8cm}{Вычисленных матриц Гессе}& \multirow{2}{2,25cm}{5}&  \multirow{2}{2,25cm}{2}&
	\multirow{2}{2,25cm}{4}
	\\ & & &   \\
	\hline
		
		
	\end{tabular}
	\centering{}
	\caption{\centering{Результаты вычислений для функции Розенброка $f_2(x,y)$ для методов внутренних и внешних штрафных функций на допустимом множествe $A$ в зависимости от начальной точки}}
\end{table}

\begin{table}[!h]
	\centering
	\begin{tabular}[c]{|c|c|c|c|}
		\hline
		\multirow{2}{2.8cm}{Начальная точка}& \multirow{2}{2.7cm}{$[-2,2]$}&  \multirow{2}{2,7cm}{$[2,4]$}&
		\multirow{2}{2,7cm}{$[2,2]$}
		\\ & & & 
		\\\hline Количество & \multicolumn{3}{c|}{Метод внутренних штрафных функций на $B$} \\ \hline  
		\multirow{2}{2.8cm}{Итераций}& \multirow{2}{2,25cm}{11}&  \multirow{2}{2,25cm}{11}&
		\multirow{2}{2,25cm}{11}
		\\ & & &    \\ \hline
		\multirow{2}{2.8cm}{Вычисленных функций}& \multirow{2}{2,25cm}{12}&  \multirow{2}{2,25cm}{12}&
		\multirow{2}{2,25cm}{12}
		\\ & & &   \\ \hline
		\multirow{2}{2.8cm}{Вычисленных градиентов}& \multirow{2}{2,25cm}{19}&  \multirow{2}{2,25cm}{25}&
		\multirow{2}{2,25cm}{32}
		\\ & & &   \\ \hline
		\multirow{2}{2.8cm}{Вычисленных матриц Гессе}& \multirow{2}{2,25cm}{19}&  \multirow{2}{2,25cm}{25}&
		\multirow{2}{2,25cm}{32}
		\\ & & &   \\ 
		\hline Количество & \multicolumn{3}{c|}{Метод внешних штрафных функций на $B$} \\ \hline  
		\multirow{2}{2.8cm}{Итераций}& \multirow{2}{2,25cm}{2}&  \multirow{2}{2,25cm}{2}&
		\multirow{2}{2,25cm}{2}
		\\ & & &    \\ \hline
		\multirow{2}{2.8cm}{Вычисленных функций}& \multirow{2}{2,25cm}{3}&  \multirow{2}{2,25cm}{3}&
		\multirow{2}{2,25cm}{3}
		\\ & & &    \\ \hline
		\multirow{2}{2.8cm}{Вычисленных градиентов}& \multirow{2}{2,25cm}{6}&  \multirow{2}{2,25cm}{8}&
		\multirow{2}{2,25cm}{7}
		\\ & & &    \\ \hline
		\multirow{2}{2.8cm}{Вычисленных матриц Гессе}& \multirow{2}{2,25cm}{6}&  \multirow{2}{2,25cm}{8}&
		\multirow{2}{2,25cm}{7}
		\\ & & &   \\ \hline
		
		
	\end{tabular}
	\centering{}
	\caption{\centering{Результаты вычислений для функции Розенброка $f_2(x,y)$ для методов внутренних и внешних штрафных функций на допустимом множествe $B$ в зависимости от начальной точки}}
\end{table}

\clearpage
\begin{figure}[!h]
	\begin{minipage}[h]{140pt}
	%	\center{\includegraphics[width=140pt]{a1.pdf}} \\а) 
	\end{minipage}
	\hfill
\begin{minipage}[h]{140pt}
	%\center{\includegraphics[width=140pt]{a2.pdf}} \\б) 
\end{minipage}
	\hfill
\begin{minipage}[h]{140pt}
%	\center{\includegraphics[width=140pt]{a3.pdf}} \\в) 
\end{minipage}
	\vfill
	$\,$
	\centering{ }
	\caption{\centering{Визуализация метода внутренних штрафных функций на допустимом множествe $A$ при $\varepsilon=0.01$ для а) квадратичной функции $f_1(x,y)$, б) функции Розенброка $f_2(x,y)$ в) функции Розенброка $f_3(x,y)$}}
\end{figure}

\begin{figure}[!h]
	\begin{minipage}[h]{140pt}
	%	\center{\includegraphics[width=140pt]{b1.pdf}} \\а) 
	\end{minipage}
	\hfill
	\begin{minipage}[h]{140pt}
	%	\center{\includegraphics[width=140pt]{b2.pdf}} \\б) 
	\end{minipage}
	\hfill
	\begin{minipage}[h]{140pt}
	%	\center{\includegraphics[width=140pt]{b3.pdf}} \\в) 
	\end{minipage}
	\vfill
	$\,$
	\centering{ }
	\caption{\centering{Визуализация метода внутренних штрафных функций на допустимом множествe $A$ при $\varepsilon=0.000001$ для а) квадратичной функции $f_1(x,y)$, б) функции Розенброка $f_2(x,y)$ в) функции Розенброка $f_3(x,y)$}}
\end{figure}

\begin{figure}[!h]
	\begin{minipage}[h]{140pt}
	%	\center{\includegraphics[width=140pt]{c1.pdf}} \\а) 
	\end{minipage}
	\hfill
	\begin{minipage}[h]{140pt}
	%	\center{\includegraphics[width=140pt]{c2.pdf}} \\б) 
	\end{minipage}
	\hfill
	\begin{minipage}[h]{140pt}
	%	\center{\includegraphics[width=140pt]{c3.pdf}} \\в) 
	\end{minipage}
	\vfill
	$\,$
	\centering{ }
	\caption{\centering{Визуализация метода внутренних штрафных функций на допустимом множествe $B$ при $\varepsilon=0.01$ для а) квадратичной функции $f_1(x,y)$, б) функции Розенброка $f_2(x,y)$ в) функции Розенброка $f_3(x,y)$}}
\end{figure}

\begin{figure}[!h]
	\begin{minipage}[h]{140pt}
	%	\center{\includegraphics[width=140pt]{d1.pdf}} \\а) 
	\end{minipage}
	\hfill
	\begin{minipage}[h]{140pt}
	%	\center{\includegraphics[width=140pt]{d2.pdf}} \\б) 
	\end{minipage}
	\hfill
	\begin{minipage}[h]{140pt}
	%	\center{\includegraphics[width=140pt]{d3.pdf}} \\в) 
	\end{minipage}
	\vfill
	$\,$
	\centering{ }
	\caption{\centering{Визуализация метода внутренних штрафных функций на допустимом множествe $B$ при $\varepsilon=0.000001$ для а) квадратичной функции $f_1(x,y)$, б) функции Розенброка $f_2(x,y)$ в) функции Розенброка $f_3(x,y)$}}
\end{figure}

	\begin{figure}[!h]
		\begin{minipage}[h]{140pt}
		%	\center{\includegraphics[width=140pt]{e1.pdf}} \\а) 
		\end{minipage}
		\hfill
		\begin{minipage}[h]{140pt}
		%	\center{\includegraphics[width=140pt]{e2.pdf}} \\б) 
		\end{minipage}
		\hfill
		\begin{minipage}[h]{140pt}
		%	\center{\includegraphics[width=140pt]{e3.pdf}} \\в) 
		\end{minipage}
		\vfill
		$\,$
		\centering{ }
		\caption{\centering{Визуализация метода внeшних штрафных функций на допустимом множествe $A$ при $\varepsilon=0.01$ для а) квадратичной функции $f_1(x,y)$, б) функции Розенброка $f_2(x,y)$ в) функции Розенброка $f_3(x,y)$}}
	\end{figure}
	
	\begin{figure}[!h]
		\begin{minipage}[h]{140pt}
		%	\center{\includegraphics[width=140pt]{f1.pdf}} \\а) 
		\end{minipage}
		\hfill
		\begin{minipage}[h]{140pt}
		%	\center{\includegraphics[width=140pt]{f2.pdf}} \\б) 
		\end{minipage}
		\hfill
		\begin{minipage}[h]{140pt}
		%	\center{\includegraphics[width=140pt]{f3.pdf}} \\в) 
		\end{minipage}
		\vfill
		$\,$
		\centering{ }
		\caption{\centering{Визуализация метода внeшних штрафных функций на допустимом множествe $A$ при $\varepsilon=0.000001$ для а) квадратичной функции $f_1(x,y)$, б) функции Розенброка $f_2(x,y)$ в) функции Розенброка $f_3(x,y)$}}
	\end{figure}
	
	\begin{figure}[!h]
		\begin{minipage}[h]{140pt}
		%	\center{\includegraphics[width=140pt]{g2.pdf}} \\б) 
		\end{minipage}
		\hfill
		\begin{minipage}[h]{140pt}
		%	\center{\includegraphics[width=140pt]{g3.pdf}} \\в) 
		\end{minipage}
		\vfill
		$\,$
		\centering{ }
		\caption{\centering{Визуализация метода внeшних штрафных функций на допустимом множествe $B$ при $\varepsilon=0.01$ для а) квадратичной функции $f_1(x,y)$, б) функции Розенброка $f_2(x,y)$ в) функции Розенброка $f_3(x,y)$}}
	\end{figure}
	
	\begin{figure}[!h]
		\begin{minipage}[h]{140pt}
		%	\center{\includegraphics[width=140pt]{h1.pdf}} \\а) 
		\end{minipage}
		\hfill
		\begin{minipage}[h]{140pt}
		%	\center{\includegraphics[width=140pt]{h2.pdf}} \\б) 
		\end{minipage}
		\hfill
		\begin{minipage}[h]{140pt}
		%	\center{\includegraphics[width=140pt]{h3.pdf}} \\в) 
		\end{minipage}
		\vfill
		$\,$
		\centering{ }
		\caption{\centering{Визуализация метода внeшних штрафных функций на допустимом множествe $B$ при $\varepsilon=0.000001$ для а) квадратичной функции $f_1(x,y)$, б) функции Розенброка $f_2(x,y)$ в) функции Розенброка $f_3(x,y)$}}
\end{figure}



\clearpage
\subsection {Выводы}

В результате выполнения лабораторной работы были реализованы два метода:
\begin{itemize}
	\item Метод внутренних штрафных функций (барьерных функций);
	\item Метод внешних штрафных функций.
\end{itemize}

Во всех методах с заранее заданной точностью были получены точка минимума и минимальное значение в этой точке. 
В случае нахождения точки минимума функции внутри допустимого множества поиска алгоритмы быстро к ней сходятся. Метод внешних штрафных функций сходится даже быстрее, так как исследуемая функция не меняется в допустимой области. В случае нахождения точки минимума функции за границей допустимого множества алгоритмы сходятся медленнее к наименьшей точке на границе. При этом метод внутренних штрафных функций сходится изнутри области, а внешних — снаружи. Первая итерация методов делает большой шаг в сторону точки минимума, а следующие уточняют её положение с изменением штрафной функции. 

\newpage
\begin{thebibliography}{3}
	\bibitem{first} Аттетков А. В. Методы оптимизации: Учеб. для вузов / А. В. Аттетков, С. В. Галкин, В.С. Зарубин –  М.: Изд-во МГТУ им. Н.Э. Баумана, 2003. – 440 с.
	
	
\end{thebibliography}
\end{document} 